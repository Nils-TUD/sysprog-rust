\section{Getting Started}

\begin{frame}[fragile]
    \frametitle{Installation}

    \begin{itemize}
        \item Please install the latest stable version of Rust
        \item Primary way: \texttt{rustup} (installer and version management)
        \item Some distributions (e.g., Arch) have a package for Rust or \texttt{rustup}
        \item Otherwise:
\begin{lstlisting}[language=bash]
$ curl --proto '=https' --tlsv1.2 https://sh.rustup.rs -sSf > rup.sh
# check if it's safe and use a fresh shell
$ sh rup.sh
\end{lstlisting}
    \end{itemize}
\end{frame}

\begin{frame}
    \frametitle{Overview}

    \begin{itemize}
        \item \texttt{rustc} is the Rust compiler; almost never invoked by the user
        \item \texttt{cargo} is Rust's build system and package manager
        \begin{itemize}
            \item Cargo.toml describes what to build and its dependencies
            \item \texttt{cargo} downloads dependencies and builds everything automatically
            \item Every library/application is a \emph{crate}
            \item Crates can be found on \url{https://crates.io} (or \url{https://lib.rs})
        \end{itemize}
    \end{itemize}
\end{frame}

\begin{frame}[fragile]
    \frametitle{Let's Build Hello World!}

    \begin{lstlisting}[language=bash]
    $ cargo new hello
    $ cd hello
    $ cargo run
    \end{lstlisting}
\end{frame}
