 \section{Generics, Traits, and Error Handling}

\begin{frame}[fragile]
    \frametitle{Basics of Generics}

    \begin{itemize}
        \item Generics allow to define functions/structs/enums for a variety of concrete types:
        \begin{lstlisting}[language=rust]
fn foo<T>(arg: T) { /* ... */ }
        \end{lstlisting}
        \item Generics have no runtime overhead due to monomorphization:
        \begin{lstlisting}[language=rust]
fn foo<T>(arg: T) { /* ... */ }
// is compiled to something like:
fn foo_u32(arg: u32) { /* ... */ }
fn foo_u64(arg: u64) { /* ... */ }
        \end{lstlisting}
        \item Unlike C++, Rust is strict about the requirements for type parameters\\
        (based on traits, as we will see shortly)
    \end{itemize}
\end{frame}

\begin{frame}[fragile]
    \frametitle{Generic Types}

    \begin{columns}
    \begin{column}{0.5\textwidth}

    \begin{itemize}
        \item<1-> Generic function
        \begin{lstlisting}[language=rust]
fn head<T>(elems: &Vec<T>) -> &T {
    &elems[0]
}
assert_eq!(*head(&vec![1, 2]), 1);
        \end{lstlisting}

        \item<2-> Generic struct
        \begin{lstlisting}[language=rust]
struct Rectangle<T> {
    width: T,
    height: T,
}
Rectangle { width: 1.2, height: 4.5 }
        \end{lstlisting}
    \end{itemize}

    \end{column}
    \begin{column}{0.5\textwidth}

    \begin{itemize}
        \item<3-> Generic enum
        \begin{lstlisting}[language=rust]
enum Option<T> {
    Some(T),
    None,
}
        \end{lstlisting}

        \item<4-> Generic method
        \begin{lstlisting}[language=rust]
impl<T: AddAssign> Rectangle<T> {
    fn widen(&mut self, amount: T) {
        self.width += amount;
    }
}
        \end{lstlisting}
    \end{itemize}

    \end{column}
    \end{columns}
\end{frame}

\begin{frame}[fragile]
    \frametitle{Trait Basics}

    \begin{itemize}
        \item A \emph{trait} defines a behavior that can be implemented by multiple types:
        \begin{lstlisting}[language=rust]
trait Shape {
    fn area(&self) -> u32;
}
        \end{lstlisting}

        \pause

        \item Implementing a trait for a type:
        \begin{lstlisting}[language=rust]
impl Shape for Rectangle {
    fn area(&self) -> u32 {
        self.width * self.height
    }
}
        \end{lstlisting}
    \end{itemize}
\end{frame}

\begin{frame}[fragile]
    \frametitle{More on Traits (1)}

    \begin{itemize}
        \item Using trait bounds:
        \begin{lstlisting}[language=rust]
fn sum<T: AddAssign + Copy + Default>(nums: &Vec<T>) -> T {
    let mut sum = T::default();
    for n in nums { sum += *n; }
    sum
}
        \end{lstlisting}

        \pause

        \item Static vs. dynamic dispatch:
        \begin{lstlisting}[language=rust]
// one function for each type
fn static_dispatch<T: Shape>(sh: &T) { }
fn static_dispatch(sh: &impl Shape) { } // syntactic sugar
// one function for all types, dispatched at runtime
fn dynamic_dispatch(sh: &dyn Shape) { }
        \end{lstlisting}
    \end{itemize}
\end{frame}

\begin{frame}[fragile]
    \frametitle{More on Traits (2)}

    \begin{itemize}
        \item Derive attribute:
        \begin{lstlisting}[language=rust]
#[derive(Debug)]
struct Point {
    x: u32,
    y: u32,
}
        \end{lstlisting}

        \pause

        \begin{lstlisting}[language=rust]
let p = Point { x: 0, y: 16 };
println!("p = {:?}", p); // prints "p = Point { x: 0, y: 16 }"
        \end{lstlisting}
    \end{itemize}
\end{frame}

\begin{frame}[fragile]
    \frametitle{Copy vs. Move Semantics}

    \begin{block}{C++}
        \begin{itemize}
            \item Copy semantics by default
            \item Copy constructor etc. is auto-implemented by compiler (opt out possible)
            \item Programmer can opt into move semantics by implementing move constructor etc.
        \end{itemize}
    \end{block}

    \pause

    \begin{block}{Rust}
        \begin{itemize}
            \item Move semantics by default: ownership is transferred
            \item Programmer can opt into copy semantics via \texttt{\#[derive(Copy)]}
            \item If a type implements \texttt{Copy}, a flat copy is performed instead of ownership transfer
            \item Deep copies are explicit via \texttt{clone} (see \texttt{Clone} trait)
        \end{itemize}
    \end{block}
\end{frame}

\begin{frame}[fragile]
    \frametitle{Error Handling}

    \begin{itemize}
        \item Unrecoverable errors with \texttt{panic!}:
        \begin{itemize}
            \item Sometimes the best you can do
            \item Can perform stack unwinding or not (set panic=abort)
            \item Provides a backtrace to the user
        \end{itemize}

        \pause

        \item Recoverable errors with \texttt{Result}:
        \begin{lstlisting}[language=rust]
enum Result<T, E> {
    Ok(T),
    Err(E),
}
        \end{lstlisting}
    \end{itemize}
\end{frame}

\begin{frame}[fragile]
    \frametitle{Error Handling Basics}

    \begin{exampleblock}{Returning errors (simplified \texttt{std::fs::File::open})}
    \begin{lstlisting}[language=rust]
    pub fn open(path: &str) -> Result<File, Error> {
        ...
        if ... { return Err(Error::NotFound); }
        ...
    }
    \end{lstlisting}
    \end{exampleblock}

    \pause

    \begin{exampleblock}{Handling errors}
    \begin{lstlisting}[language=rust]
    let mut file = std::fs::File::open("myfile.txt").expect("open failed");
    \end{lstlisting}
    \end{exampleblock}
\end{frame}

\begin{frame}[fragile]
    \frametitle{Passing Errors Upwards}

    \begin{lstlisting}[language=rust]
    let mut file = std::fs::File::open(path)?;
    // is equivalent to:
    let mut file = match std::fs::File::open(path) {
        Ok(file) => file,
        Err(e) => return Err(e),
    };
    \end{lstlisting}

    \pause

    \begin{lstlisting}[language=rust]
    fn read_file(path: &str) -> Result<String, Error> {
        let mut file = std::fs::File::open(path)?;
        let mut s = String::new();
        file.read_to_string(&mut s)?;
        Ok(s)
    }
    \end{lstlisting}
\end{frame}

\begin{frame}[fragile]
    \frametitle{Option Instead of Nullpointers}

    \begin{itemize}
        \item Similar to \texttt{Result} for errors, Rust uses \texttt{Option} for optional values:
        \begin{lstlisting}[language=rust]
enum Option<T> {
    Some(T),
    None,
}
        \end{lstlisting}

    \pause

        \item Important methods on \texttt{Result} and \texttt{Option}
        \begin{itemize}
            \item \texttt{unwrap}: panic if \texttt{None/Err}
            \item \texttt{expect}: panic with message if \texttt{None/Err}
            \item \texttt{*\_or\_else}: transformation
        \end{itemize}

    \pause

        \item More at \url{https://doc.rust-lang.org/stable}
    \end{itemize}
\end{frame}

\begin{frame}[fragile]
    \frametitle{Exercise 3 -- Proper Error Handling}

    \begin{itemize}
        \item Let's add proper error handling to our books collection
        \item Use \texttt{Result} and \texttt{Option} where appropriate
        \item Get rid of all panics/unwraps
        \item Hints:
        \begin{itemize}
            \item Introduce your own error enum
            \item Attach \texttt{\#[derive(Debug)]} to your error enum
            \item Implement \texttt{From<std::num::ParseIntError>} for your enum
            \item Implement \texttt{Display} for Book
        \end{itemize}
    \end{itemize}
\end{frame}
